\documentclass[UTF8]{ctexart}
\usepackage{amssymb}
\usepackage{amsmath}
\usepackage{xcolor}
\usepackage[normalem]{ulem} % use normalem to protect \emph
\newcommand\hl{\bgroup\markoverwith
  {\textcolor{yellow}{\rule[-.5ex]{2pt}{2.5ex}}}\ULon}
\newtheorem{theorem}{\hspace{2em}定理}[section]
\newtheorem{definition}{\hspace{2em}定义}[section]
\newtheorem{corollary}{\hspace{2em}推论}[section]
\newtheorem{proof}{\hspace{2em}证明}[section]

\begin{document}
    \begin{itemize}
        \renewcommand{\labelitemi}{\scriptsize$\blacksquare$}
        \item 近代密码涉及到许多数学分支, 如数理统计、信息论、数论、有限域理论、复杂性理论、甚至于代数、几何等.
        \item 数学是近代密码学中不可或缺的工具.
    \end{itemize}

    \section{数论}

    \subsection{数的m进制表示}
    \begin{enumerate}
        \item 十进制表示
        \begin{itemize}
            \renewcommand{\labelitemi}{\scriptsize$\blacksquare$}
            \item 十进制是最方便的一种整数表示法

            例如: $1987=1\times 10^3+9\times 10^2+8\times 10+7$,
            $53721=5\times 10^4+3\times 10^3+7\times 10^2+2\times 10+1$

            \item 实际上, 可以用任何进制表示一个数.
            \begin{theorem}
                设m是大于1的正整数, 则每一个正整数n可唯一表示为
                $$n=c_km^k+c_{k-1}m^{k-1}+\cdots +c_1m+c_0$$
                其中$c_j(j=0,1,2,\cdots ,k)$是整数, 且$0\le c_j < m, c_k\ne 0$. 记作: $n=(c_kc_{k-1}\cdots c_1c_0)_m$
            \end{theorem}
        \end{itemize}
        \item m进制表示的具体做法
        \begin{itemize}
            \renewcommand{\labelitemi}{\scriptsize$\blacksquare$}
            \item 将一个整数n表示为m进制时, 主要是确定$c_0, c_1, \cdots, c_{k-1}, c_{k}$
            \item $\lfloor \frac{n}{m} \rfloor$表示n除以m后, 取其整数部分(也就是比$\frac{n}{m}$小的最大整数),确定$c_0,c_1,c_2\cdots ,c_{k-1},c_k$的方法如下:
            \begin{enumerate}
                \item 令$r_1=c_0, n_0=n$, 则有
                $$n_1=\lfloor \frac{n_0}{m} \rfloor=c_km^{k-1}+c_{k-1}m^{k-2}+\cdots +c_1m+c_0$$

                \item 令$r_2=c_1$, 则有
                $$n_2=\lfloor \frac{n_1}{m} \rfloor=c_km^{k-2}+c_{k-1}m^{k-3}+\cdots +c_2m+c_1$$

                \item 令$r_3=c_2, \cdots \cdots$
                \item 若$n_i>m$, 令$r_{i+1}=c_i, i=0, 1, 2, \cdots$
                $$n_{i+1}=\lfloor \frac{n_i}{m} \rfloor=c_km^{k-i-1}+c_{k-1}m^{k-i-2}+\cdots +c_{i+2}m+c_{i+1}$$

                \item 直到
                $$n_{k+1}=\lfloor \frac{n_k}{m} \rfloor=0, c_k=r_{k+1}$$
                即$n_k<m$为止
            \end{enumerate}
        \end{itemize}
        \item 举例

        例如 $n=389, m=5$

        解 令$n_0=n=389$

        则有
        $$
        \begin{aligned}
            &n_1=\lfloor \frac{n_0}{5} \rfloor=\lfloor \frac{389}{5} \rfloor=77, r_1=4=c_0\\
            &n_2=\lfloor \frac{n_1}{5} \rfloor=\lfloor \frac{77}{5} \rfloor=15, r_2=2=c_1\\
            &n_1=\lfloor \frac{n_0}{5} \rfloor=\lfloor \frac{389}{5} \rfloor=77, r_3=0=c_2\\
            &n_1=\lfloor \frac{n_0}{5} \rfloor=\lfloor \frac{389}{5} \rfloor=77, r_4=3=c_3
        \end{aligned}
        $$
        故$389=3\times 5^3+0\times 5^2+2\times 5^1+4=(3024)_5$

        例如 $n=389, m=2$

        则有
        $$
        \begin{aligned}
            &n_1=\lfloor \frac{n_0}{2} \rfloor=\lfloor \frac{389}{2} \rfloor=194, c_0=1\\
            &n_2=\lfloor \frac{n_1}{2} \rfloor=\lfloor \frac{194}{2} \rfloor=97, c_0=0\\
            &n_3=\lfloor \frac{n_2}{2} \rfloor=\lfloor \frac{97}{2} \rfloor=48, c_0=1\\
            &n_4=\lfloor \frac{n_3}{2} \rfloor=\lfloor \frac{48}{2} \rfloor=24, c_0=0\\
            &n_5=\lfloor \frac{n_4}{2} \rfloor=\lfloor \frac{24}{2} \rfloor=12, c_0=0\\
            &n_6=\lfloor \frac{n_5}{2} \rfloor=\lfloor \frac{12}{2} \rfloor=6, c_0=0\\
            &n_7=\lfloor \frac{n_6}{2} \rfloor=\lfloor \frac{6}{2} \rfloor=3, c_0=0\\
            &n_8=\lfloor \frac{n_7}{2} \rfloor=\lfloor \frac{3}{2} \rfloor=1, c_0=1\\
            &n_9=\lfloor \frac{n_8}{2} \rfloor=\lfloor \frac{1}{2} \rfloor=0, c_0=1
        \end{aligned}
        $$
        故$389=2^8+2^7+2^2+1=(110000101)_2$
    \end{enumerate}

    \subsection{数的因数分解}
    \begin{definition}{素数}
        只能被1和其自身除尽的正整数称为

        素数$(1, 2, 3, 5, \cdots)$
    \end{definition}
    \begin{definition}{合数}
        不是1且非素数的正整数称为

        合数$(4, 6, 8, 9, 10, \cdots)$

    \end{definition}

    a除尽b表示$a|b$.

    以后不特别说明英文字母$a, b, c,\cdots$等都表示正整数.
    \begin{definition}{公因子}
        若$a|b$, 且$a|c$, 则称a是b和c的公因子
    \end{definition}
    \begin{definition}{最大公因子}
        若a是b和c的公因子, 且b和c的每一个公因子都能除尽a, 则称a是b和c的最大公因子, 表示为
        $a=gcd(b,c)$或者$a=(b,c)$
    \end{definition}
    \begin{definition}{倍数}
        若$a|c$ ,则称c是a的倍数, 若$a|c$且$b|c$, 则称c是a和b的公倍数
    \end{definition}
    \begin{definition}{最小公倍数}
        若a和b的公倍数c能除尽a和b的任意公倍数, 则称c是a和b的最小公倍数, 表示为
        $c=lcm\{a,b\}$或$c=[a,b]$
    \end{definition}

    \begin{theorem}
        若$a=bq+r$, 则$gcd\{a, b\}=\pm gcd\{b, r\}$
    \end{theorem}
    \begin{theorem}
        每一对不全为零的整数, 必有一个正的最大公因数.
    \end{theorem}

    例: 求$gcd\{726, 393\}$

    解: 辗转相除法

    $$
    \begin{aligned}
        &gcd\{726, 393\}=gcd\{393, 333\}\\
        &gcd\{393, 333\}=gcd\{333, 60\}\\
        &gcd\{60, 33\}=gcd\{33, 27\}\\
        &gcd\{33, 27\}=gcd\{27, 6\}\\
        &gcd\{27, 6\}=gcd\{6, 3\}\\
        &gcd\{6, 3\}=3
    \end{aligned}
    $$

    \begin{theorem}
        若$d=gcd\{a, b\}$, 则存在整数p和q, 使得$d=pa+qb$

        若$gcd\{a,b\}=\pm 1$, 则称a和b互素. 1和任意整数互素

        若a和b互素, 则必存在整数p和q, 使得$pa+qb=1$
    \end{theorem}

    例: 以$3=gcd\{726, 393\}$为例.

    解:
    $$
    \begin{aligned}
        3&=27-4\times 6\\
         &=27-4\times (33-27)\\
         &=-4\times 33+5\times 27\\
         &=-4\times 33+5\times (60-33)\\
         &=5\times 60-9\times 33\\
         &=5\times 60-9\times (333-5\times 60)\\
         &=-9\times 333+50\times 60\\
         &=-9\times 333+50\times (393-333)\\
         &=50\times 393-59\times 333\\
         &=50\times 393-59\times (726-393)\\
         &=-59\times 726+109\times 393
    \end{aligned}
    $$
    所以, $3\equiv -59\times 726+109\times 393$

    \subsection{同余类}
    \begin{itemize}
        \renewcommand{\labelitemi}{\scriptsize$\blacksquare$}
        \item 若$m|a-b$, 即$a-b=km$, 就称a和b模m同余, 记作: $a\equiv b \bmod m$. m被称为这个同余式的模
        \item 同余关系和通常意义的相等颇为相似, 但实质不同.
        \item a和b模m同余表示a和b除以m的余数相同, 或$a-b$是m的倍数.
    \end{itemize}

    例如 $5\equiv 2 \bmod 3, 8\equiv 2 \bmod3, 11\equiv 2 \bmod3$
    即$a\equiv b \bmod m$, 实际上是说明存在一个整数k, 使得$a\equiv km+b$

    \begin{theorem}
        模m的同余关系满足
        \begin{enumerate}
            \renewcommand\labelenumi{(\theenumi)}
            \item 自反性, 即$a\equiv a \bmod m$.
            \item 对称性, 即若$a\equiv b \bmod m$, 则$b\equiv a \bmod m$.
            \item 传递性, 即若$a\equiv b \bmod m$, $b\equiv c \bmod m$, 则$a\equiv c \bmod m$.
        \end{enumerate}
    \end{theorem}
    \begin{theorem}
        若$a\equiv b \bmod m, c\equiv d \bmod m$, 则
        \begin{enumerate}
            \renewcommand\labelenumi{(\theenumi)}
            \item $a\pm c\equiv b\pm d \bmod m$
            \item $ac\equiv bd \bmod m$
        \end{enumerate}
    \end{theorem}
    \begin{theorem}
        若$ac\equiv bc \bmod m$, 且c和m互素, 则$a\equiv b \bmod m$.
    \end{theorem}
    \begin{theorem}
        若$ac\equiv bc \bmod m, d=(c,m)$, 则$a\equiv b \bmod (m/d)$
    \end{theorem}

    例如:
    $\because$
    $
    \begin{aligned}
        &42\equiv 7\bmod 5\\
        &6\times 7\equiv 1\times 7\bmod 5
    \end{aligned}
    $

    $\therefore$ $6\equiv 1\bmod 5$

    \subsection{线性同余方程}
    \begin{itemize}
        \item 线性同余方程为: $ax\equiv b\bmod m$
        \item 若整数$x_1$满足线性同余方程, 即$ax_1\equiv b\bmod m$, 则模m与$x_1$同余的所有整数$x(x\equiv x_1\bmod m)$都满足这个线性同余方程.
        \item 若$x_2$是模m与$x_1$的同余整数, 即$x_2\equiv x_1\bmod m$, 则$ax_2\equiv b\bmod m$
    \end{itemize}

    例如: $2x\equiv 3\bmod 5, x\equiv 4\bmod 5$是这个线性同余式的解.

    \begin{theorem}
        同余式
        $$ax=b\bmod m$$
        有解的充要条件是 $d|b$, 其中$d=(a,m)$. 令$m'=\frac{m}{d}$, 若$x_0$是$ax=b\bmod m$的一个解, 则其所有解x均满足
        $$x\equiv x_0(\bmod m')$$
    \end{theorem}
    \begin{corollary}
        设$(a, m)=1$, 则同余式$ax\equiv b(\bmod m)$恰有唯一解
        $$x\equiv x_o(\bmod m)$$
    \end{corollary}

    \subsection{联立的同余式和中国剩余定理}
    \begin{theorem}
        下列两个同余式
        $$x\equiv b_1\bmod m_1, x\equiv b_2\bmod m_2$$
        有一个共同解的充要条件为
        $$b_1\equiv b_2\bmod d, d=(m_1, m_2)$$
        即$$(m_1, m_2)|(b_1, b_2)$$

        对于n个\hl{联立同余式}有类似结果
    \end{theorem}
    \begin{theorem}
        联立同余式
        $$x\equiv b_i\bmod m_i, i=1,2,\cdots, n$$
        有一个共同解的充要条件为
        $$(m_i, m_j)|(b_i, b_j), i\ne j, i, j=1,2,3,\cdots,n$$
    \end{theorem}
    \begin{theorem}
        若$a\equiv b\bmod m_1, a\equiv b\bmod m_2, \cdots, a\equiv b\bmod m_k$, 则
        $$a\equiv b\bmod [m_1, m_2, \cdots, m_k]$$
        式中$[m_1, m_2, \cdots, m_k]$表示最小公倍数.
    \end{theorem}
    \begin{definition}{中国剩余定理}
        设$m_1, m_2, \cdots, m_k$是两两互素的正整数, $M=m_1m_2\cdots m_k$, $M_i=\frac{M}{m_i}(i=1,2,\cdots,k)$,
        则同余式方程组
        $$x\equiv b_i\bmod m_i, i=1,2,\cdots, k$$
        有唯一解
        $$x\equiv b_1M_1y_1+b_2M_2y_2+\cdots+b_kM_ky_k(mod M)$$
        其中$M_iy_i\equiv 1(\bmod m_i), i=1,2,\cdots, k$.
    \end{definition}
    \begin{proof}
        令$M=m_1m_2\cdots m_k$
        $$M_j=\frac{M}{m_j}=m_1m_2\cdots m_{j-1}m_{j+1}\cdots m_k$$
        求$y_j$使得
        $$M_iy_i\equiv 1\bmod m_j, j=1,2,\cdots,k$$
        由于$(M_j, m_j)=1$, 故$y_j$存在. 令
        $$x=b_1M_1y_1+b_2M_2y_2+\cdots +b_kM_ky_k$$
    \end{proof}

    例题 求下列方程组的解
    $$
    \left\{ \begin{aligned}
        &x\equiv 1\bmod 2\\
        &x\equiv 2\bmod 3\\
        &x\equiv 3\bmod 5
    \end{aligned}\right.
    $$

    由上述方程可以得到
    $$b_1=1, b_2=2, b_3=3$$
    $$m_1=2, m_2=3, m_3=5$$

    所以
    $$
    \begin{aligned}
        &M=m_1\cdot m_2\cdot m_3=2\times 3\times 5=30\\
        &M_1=\frac{M}{m_1}=\frac{30}{2}=15\\
        &M_2=\frac{M}{m_2}=\frac{30}{3}=10\\
        &M_3=\frac{M}{m_3}=\frac{30}{5}=6\\
    \end{aligned}
    $$

    因而
    $$
    \left\{ \begin{aligned}
        &15y_1\equiv 1\bmod 2, y_1=1\\
        &10y_2\equiv 1\bmod 3, y_2=1\\
        &6y_3\equiv 1\bmod 5, y_3=1\\
    \end{aligned}\right.
    $$

    可得
    $$
    \begin{aligned}
        x&=b_1M_1y_1+b_2+M_2y_2+b_3M_3y_3\\
         &=1\times 15\times 1+2\times 10\times  1+3\times 6\times 1\\
         &=15+20+18\\
         &=53\equiv 23\bmod 30
    \end{aligned}
    $$

    例题: 求解方程
    $$
    \left\{ \begin{aligned}
        &x\equiv 0(\bmod 3)\\
        &x\equiv 1(\bmod 5)\\
        &x\equiv 2(\bmod 7)
    \end{aligned}\right.
    $$

    解: 由题目条件可知
    $$m_1=3, m_2=5, m_3=7, b_1=0, b_2=1, b_3=2$$
    故有
    $$
    \begin{aligned}
        &M=m_1\cdot m_2\cdot m_3=3\times 5\times 7=105\\
        &M_1=\frac{M}{m_1}=\frac{105}{3}=35\\
        &M_2=\frac{M}{m_2}=\frac{105}{5}=21\\
        &M_3=\frac{M}{m_3}=\frac{105}{7}=15
    \end{aligned}
    $$
    则
    $$
    \begin{aligned}
        &35y_1=1(\bmod 3), y_1=2\\
        &21y_2=1(\bmod 5), y_2=1\\
        &15y_3=1(\bmod 7), y_3=1
    \end{aligned}
    $$

    最后得到解
    $$
    \begin{aligned}
        x&\equiv y_1M_1b_1+y_2M_2b_2+y_3M_3b_3(\bmod M)\\
         &\equiv [2\times 35\times 0+1\times 21\times 1+1\times 15\times 2](\bmod 105)\\
         &\equiv 51(\bmod 105)\\
    \end{aligned}
    $$

\section{公钥密码}



\end{document}