\documentclass[UTF8]{ctexart}
\usepackage{amssymb}
\usepackage{multirow}
\usepackage{amsmath}
\usepackage{xcolor}
\usepackage[normalem]{ulem} % use normalem to protect \emph
\newcommand\hl{\bgroup\markoverwith
  {\textcolor{yellow}{\rule[-.5ex]{2pt}{2.5ex}}}\ULon}

\begin{document}
\section{传统密码简介}
例题 如果用数字(0-25)分别和字母a, b, c, d,..., y, z相对应, 即

    \begin{tabular}{|c|c|c|c|c|c|c|c|c|c|c|c|c|c|} %l(left)居左显示 r(right)居右显示 c居中显示
        \hline
        明文& a& b& c& d& e& f& g& h& i& j& k& l& m\\
        \hline
        数字& 0& 1& 2& 3& 4& 5& 6& 7& 8& 9& 10& 11& 12\\
        \hline
        $k=3$密文& D& E& F& G& H& I& J& K& L& M& N& O& P\\
        \hline
        \hline
        明文& n& o& p& q& r& s& t& u& v& w& x& y& z\\
        \hline
        数字& 13& 14& 15& 16& 17& 18& 19& 20& 21& 22& 23& 24& 25\\
        \hline
        $k=3$密文& Q& R& S& T& U& V& W& X& Y& Z& A& B& C\\
        \hline
    \end{tabular}

    则密文字母$\phi$ 可以用明文字母$\theta$ 表示如下:
    \begin{equation}
        \phi = \theta + 3(\bmod26)
    \end{equation}

    若明文字母为y, 即$\theta = 24$时,
    $$\phi \equiv 24 + 3(\bmod26) \equiv 27(\bmod26) \equiv 1(\bmod26)$$

    因此, 密文字母为E.


    \begin{itemize}
        \renewcommand{\labelitemi}{\scriptsize$\blacksquare$}
        \item 公式(1)是\hl{凯撒密码}的数学形式, 也表示一种算法.式中\hl{密钥}为3, 实际表示倒退3步.
        \item 此例中, 密钥可为$1~ 25$之间的任意一个数字, 但若是选择0为密钥, 则密钥=明文, 实际上没有加密. 故式子(1)可以推广成
        $$\phi = \theta + k(\bmod 26)$$

        这里$k\in K, K={1, 2, 3,..., 24, 25}$是密钥集合或密钥空间.

        \item 密码系统的两个基本单元是算法和密钥.
        \begin{itemize}
            \item 算法---是一些公式、法则或者程序, 规定着明文和密钥之间的变换方法. 算法相对是稳定的, 可是为常量.
            \item 密钥---可看成式算法中的参数, 是一个变量. 可以根据事前的规定好的安排, 或用过若干次后改变一个密钥, 或每过一段时间更换一次密钥, 等等. 为密码系统的安全, 频繁更换密钥是必要的.
        \end{itemize}
        \item 一般来讲, 算法往往不能保密, 真正需要保密的是密钥.
        \item 传统密码可以分为两大类: 换位密码和代替密码, 有时可能是二者的组合
    \end{itemize}

    \section{换位密码}
    \begin{itemize}
        \renewcommand{\labelitemi}{\scriptsize$\blacksquare$}
        \item 换位(置换)密码在编制时, 是把明文中的字母重新排列, 但是字母本身不变, 只改变其位置.
        \item 最简单的换位密码是把明文顺序倒过来, 然后裁成固定长度的字母作为密文.

        例如 明文为this cryptosystem is not secure. (27个字符)

            密文为ERUC ESTO NSIM ETSY SOTP YRCS IHT

        \item 另一种换位密码是把明文按某一顺序排成一个矩阵, 然后按某一顺序选出矩阵中的字母已形成密文, 最后截成固定长度的字母组.

        例如 明文为this cryptosystem is not secure. (27个字符)

        解 (1) 排成5*6矩阵

        {
            \centering
            \begin{tabular}{|c|c|c|c|c|c|} %l(left)居左显示 r(right)居右显示 c居中显示
                \hline
                t& h& i& s& c& r\\
                \hline
                y& p& t& o& s& y\\
                \hline
                s& t& e& m& i& s\\
                \hline
                n& o& t& s& e& c\\
                \hline
                u& r& e& & & \\
                \hline
            \end{tabular}
        }

        (2) 选出顺序, 排列
        (3) 按字符长度5写出密文
        TYSNU HPTOR ITETE SOMSC SIERY SC

        \item 可见, 改变\hl{矩阵的大小}和\hl{选出顺序}可以得到不同形式的密码.
        \item 换位密码比较简单, 经不起"已知明文的攻击". 但是, 将其与其密码相结合, 可以得到十分有效的密码.

        例如 明文can you understand, 使用type为密钥进行列换位加密.
        解 将明文写成如下4*4矩阵

        {
            \centering
            \begin{tabular}{|c|c|c|c|c|}
                \hline
                密钥& t& y& p& e\\
                \hline
                顺序& 3& 4& 2& 1\\
                \hline
                \multirow{4}*{明文矩阵}& c& a& n& y\\
                ~ &o &u &u &n \\
                ~ &d &e &r &s \\
                ~ &t &a &n &d \\
                \hline
            \end{tabular}
        }

        例如 明文为can you understand, 试写出周期为4的\hl{周期换位密文}. 密钥是$i=1,2,3,4$的置换$f(i)=3,4,2,1$

        将明文写成如下4*4矩阵

        {
            \centering
            \begin{tabular}{|c|c|c|c|c|}
                \hline
                置换$f(i)$ &3 &4 &2 &1\\
                \hline
                \multirow{4}*{明文矩阵} &c &a &n &y\\
                ~ &o &u &u &n\\
                ~ &d &e &r &s\\
                ~ &t &a &n &d\\
                \hline
            \end{tabular}
        }

        按照密钥所确定的顺序, 按行置换该矩阵的字母并依次写出, 即得到密文

        YNC ANU OUSRDEDNTA
    \end{itemize}

    \section{代替密码}
    代替密码共有4种: 单表代替密码(monalphabetic substitution cipher)、同音代替密码(homophnic substitution cipher)、多表代替密码(polyalphabetic substitution cipher)和
    多字母代替密码(polygram substitution cipher)

    \begin{itemize}
        \item 单表代替密码: 将明文中的每个字母用密文中对应的字母取代, 在全部信息加密过程中, 明文字母与密文字母一一映射.
        \item 同音代替密码: 明文字符与密文字符是一对多映射, 每个明文字符可以变换成不同的密文字符.
        \item 多表代替密码: 用多个映射把明文字符转换为密文字符, 其中每个映射相当于单表代替密码中的一一映射.
        \item 多字母代替密码: 是最一般的, 可以对一组字符进行任何方式的代替.
    \end{itemize}

    \subsection{单表代替密码}
    构造一个密文字母表, 然后用密文字母表中的字母或字母表中来代替明文字母表的字母或字母组, 各字母或字母组的相对位置不变, 但其本身改变了, 这样的编成的密码称为单表代替密码.

    设$A={a_0, a_1, a_2, ..., a_{n-1}}$为含有n个字母的明文字母表,

    $B={b_0, b_1, b_2, ..., b_{n-1}}$是含有n个字母的密文字母表

    定义一个由A到B的映射:
    $$f: A \rightarrow B, f(a_i)=b_i$$

    若明文为: $M=(m_0, m_1, m_2, ..., m_{c-1})$,

    则相应的密文为: $C=(f(m_0), f(m_1), f(m_2), ..., f(m_{c-1}))$.

    可见, 单表代替密码的密钥就是映射f或密文字母表B.

    \subsubsection{几种典型的单表代替密码}
    \begin{itemize}
        \item 加法密码
        定义映射
        $$f(a_i) = a_j, j\equiv i+k(\bmod n) 0<k<n$$

        加密算法实际上是每一个字母向前\hl{推移}k位, 不同的k可得到不同的密文, 若令26个字母分别对应于整数0\~25, 如下表

        \begin{tabular}{|c|c|c|c|c|c|c|c|c|c|c|c|c|c|}
            \hline
            明文 &a &b &c &d &e &f &g &h &i &j &k &l &m\\
            \hline
            数字 &0 &1 &2 &3 &4 &5 &6 &7 &8 &9 &10 &11 &12\\
            \hline
            \hline
            明文 &n &o &p &q &r &s &t &u &v &w &x &y &z\\
            \hline
            数字 &13 &14 &15 &16 &17 &18 &19 &20 &21 &22 &23 &24 &25\\
            \hline
        \end{tabular}
        则加法密码辩护那实际是
        $$c\equiv m+k(\bmod 26), 0<k<26$$

        m是明文对应的\hl{明文数据}, c是密文对应的\hl{密文数据}, k是加密用的参数, 也称为\hl{密钥}

        例题 明文为data security
        解 对应的数据序列为 3 0 19 0 18 4 2 20 17 8 19 24
        k=5时的密文数据序列为 8 5 24 5 23 9 7 25 22 13 24 3
        对应的密文为          I F Y F X J H Z W N Y D
        密文数据的计算如下
        $$
        \begin{aligned}
            &c_1=m_1+5(\bmod 26)=8(\bmod 26)=8\\
            &c_2=m_2+5(\bmod 26)=5(\bmod 26)=5\\
            &c_3=m_3+5(\bmod 26)=24(\bmod 26)=24\\
            &c_4=m_4+5(\bmod 26)=5(\bmod 26)=5\\
            &c_5=m_5+5(\bmod 26)=23(\bmod 26)=23\\
            &c_6=m_6+5(\bmod 26)=9(\bmod 26)=9\\
            &c_7=m_7+5(\bmod 26)=7(\bmod 26)=7\\
            &c_8=m_8+5(\bmod 26)=25(\bmod 26)=25\\
            &c_9=m_9+5(\bmod 26)=22(\bmod 26)=22\\
            &c_{10}=m_{10}+5(\bmod 26)=13(\bmod 26)=13\\
            &c_{11}=m_{11}+5(\bmod 26)=24(\bmod 26)=24\\
            &c_{12}=m_{12}+5(\bmod 26)=29(\bmod 26)=3
        \end{aligned}
        $$

        \item 乘法密码
        映射
        $$f(a_i)=a_j, j\equiv ik(\bmod n) (k, n)=1$$

        若令26个字母与整数$0~25$对应, 则乘法密码变换实际是
        $$c\equiv (\bmod 26) 0<k<26$$

        m是明文对应的\hl{明文数据}, c是密文对应的\hl{密文数据}, k是加密用的参数, 也称为\hl{密钥}

        例子 明文为data security
        解 对应的数据序列为 3 0 19 0 18 4 2 20 17 8 19 24
        k=5时对应的密文数据序列 15 0 17 12 20 10 22 7 14 17 16
        对应的密文为 P A R A M U K W H O R Q
        密文数据的计算如下:

        $$
        \begin{aligned}
            &c_1=3\times 5=15(\bmod 26)=15\\
            &c_2=0\times 5=15(\bmod 26)=0\\
            &c_3=19\times 5=95(\bmod 26)=17\\
            &c_4=0\times 5=0(\bmod 26)=0\\
            &c_5=18\times 5=15(\bmod 26)=15\\
            &c_6=4\times 5=20(\bmod 26)=20\\
            &c_7=2\times 5=10(\bmod 26)=10\\
            &c_8=20\times 5=100(\bmod 26)=22\\
            &c_9=17\times 5=85(\bmod 26)=7\\
            &c_10=8\times 5=40(\bmod 26)=14\\
            &c_11=19\times 5=95(\bmod 26)=17\\
            &c_12=24\times 5=120(\bmod 26)=16
        \end{aligned}
        $$

        \item 仿射密码
        乘法密码与加法密码相结合便构成仿射密码

        映射 $f(a_i)a_j, j\equiv ik_1+k_0(\bmod n) (k, n)=1, 0<k_0<n$

        若令26个字母与整数0~25相对应, 则仿射密码变换实际是
        $$c\equiv mk_1 + k_0(\bmod 26) (k_1, 26)=1 0<k_0<26$$, 其中$(k_1, 26)=1$表示$k_1$与26的最大公约数为1

        m是明文对应的\hl{明文数据}, c是密文对应的\hl{密文数据}, $k_1$, $k_0$是加密的参数, 也称为\hl{密钥}.

        例如 明文为data security

        解 对应的数据序列为 3 0 19 0 18 4 2 20 17 8 19 24

        $\left. \begin{array}{c}{k_0=3}\\ {k_1=5} \end{array}\right \}$
        时的密文数据序列为 18 3 20 3 15 2313 25 10 17 20 19
        对应的密文为 S D U D P X N Z K R U T

        密文数据计算如下
        $$
        \begin{aligned}
            &c_1=m_1\times k_1 + k_0(\bmod 26)=3\times 5 + 3(\bmod 26)=18(\bmod26)=18\\
            &c_2=m_2\times k_1 + k_0(\bmod 26)=0\times 5 + 3(\bmod 26)=3(\bmod26)=3\\
            &c_3=m_3\times k_1 + k_0(\bmod 26)=19\times 5 + 3(\bmod 26)=98(\bmod26)=20\\
            &c_4=m_4\times k_1 + k_0(\bmod 26)=0\times 5 + 3(\bmod 26)=3(\bmod26)=3\\
            &c_5=m_5\times k_1 + k_0(\bmod 26)=18\times 5 + 3(\bmod 26)=93(\bmod26)15\\
            &c_6=m_6\times k_1 + k_0(\bmod 26)=4\times 5 + 3(\bmod 26)=23(\bmod26)=23\\
            &c_7=m_7\times k_1 + k_0(\bmod 26)=2\times 5 + 3(\bmod 26)=13(\bmod26)=13\\
            &c_8=m_8\times k_1 + k_0(\bmod 26)=20\times 5 + 3(\bmod 26)=103(\bmod26)=25\\
            &c_9=m_9\times k_1 + k_0(\bmod 26)=17\times 5 + 3(\bmod 26)=88(\bmod26)=10\\
            &c_10=m_10\times k_1 + k_0(\bmod 26)=8\times 5 + 3(\bmod 26)=43(\bmod26)=17\\
            &c_11=m_11\times k_1 + k_0(\bmod 26)=19\times 5 + 3(\bmod 26)=98(\bmod26)=20\\
            &c_12=m_1\times k_1 + k_0(\bmod 26)=24\times 5 + 3(\bmod 26)=123(\bmod26)=19
        \end{aligned}
        $$

        \item 多项式密码
        映射
        $$f(a_i)=a_j, j\equiv i^tk_t + i^{t-1}k_{t-1} + \cdots + ik_1 + k_0(\bmod n)$$

        其中$(k_i, n)=1, i=1,\cdots, t, 0<k_0<n$.

        若令26个字母与整数0\~{25}对应,则多项式密码变换实际是
        $$c\equiv m^tk_t + m^{t-1}k_{t-1} + \cdots + k_1m + k_0(\bmod 26)$$
        $(k_i, 26)=1, i=1, 2, \cdots, t, 0<k_0<26$

        m是明文对应的\hl{明文数据}, c是与密文对应的\hl{密文数据}, $k_t, \cdots, k_1, k_0$是加密用的参数, 也称为密钥.

        \item 密钥词组代替密码
        用一个词组或者短语作密钥, 去掉密钥中的重复字母, 把结果作为矩阵的第一行, 其次从明文字母表中补入其余字母, 最后按某一顺序从矩阵中取出字母构成密文字母表.

        如, 设密钥词组为red star, 明文字母表和密文字母表均由26个英文字母构成, 则构成矩阵如下:

        \begin{tabular}{|c|c|c|c|c|c|}
            \hline
            r& e& d& s& t& a\\
            \hline
            b& c& f& g& h& i\\
            \hline
            j& k& l& m& n& o\\
            \hline
            p& q& u& v& w& x\\
            \hline
            y& z&  &  &  &\\
            \hline
        \end{tabular}

        若选出顺序按列, 则得到明文字母和密文字母对应如下:

        \begin{tabular}{|c|c|c|c|c|c|c|c|c|c|c|c|c|c|}
            \hline
            明文字母& a& b& c& d& e& f& g& h& i& j& k& l& m\\
            \hline
            密文字母& R& B& J& P& Y& E& C& K& Q& Z& D& F& L\\
            \hline
            \hline
            明文字母& n& o& p& q& r& s& t& u& v& w& x& y& z\\
            \hline
            密文字母& U& S& G& M& V& T& H& N& W& A& I& O& X\\
            \hline
        \end{tabular}

        例如 若明文 data security

        则对应的密文为PRHR TYJNVQHO

        \item 单表代替密码的统计分析
        \begin{itemize}
            \item 加法密码和乘法密码的密钥量比较小, 可利用穷举密钥的方法进行译.
            \item 仿射密码和多项式密码的密钥量可以百、千位计, 可利用计算机进行穷举密钥的方法来破译
            \item 单表代替密码的密文字母表实质上就是明文字母表的一种排列. 对以英文字母作为明文的情况, 密文字母表可能的排列$26!\approx 4\times 10^{26}$, 所以即使使用计算机穷举密钥的方法破译也是不可能的
            \item 但是, 穷举密钥的方法不是攻击密码的唯一方法. 因为, 一旦信息足够长, 密码分析者便可利用统计分析的方法进行攻击.
            \item 任何自然语言都有许多固有的统计特性, 如果明文语言的这种统计特性在密文中有所反映, 则密码分析便可以通过分析明文和密文的统计规律破译密码
            \item 当所统计的文献篇幅足够长, 且不特别专业化, 可发现各个英文字母出现的相对频率十分稳定.
            \begin{itemize}
                \item 极高频率字母组: e
                \item 次高频率字母组: t a o n i r s h
                \item 中等频率字母组: d l u c m
                \item 低频率字母组: p f y w g b v
                \item 甚低频率字母组: j k q x z
            \end{itemize}

            单个英文字母出现频率的顺序为:

            \begin{tabular}{|c|c|c|c|c|c|c|c|c|c|c|c|c|c|}
                \hline
                序号 &1 &2 &3 &4 &5 &6 &7 &8 &9 &10 &11 &12 &13\\
                \hline
                字母 &e &t &a &o &n &r &i &s &h &d &l &f &c\\
                \hline
                \hline
                序号 &14 &15 &16 &17 &18 &19 &20 &21 &22 &23 &24 &25 &26\\
                \hline
                字母 &m &u &g &p &y &w &b &v &k &x &j &q &z\\
                \hline
            \end{tabular}
        \end{itemize}


        分析过程如下:
        \begin{enumerate}
            \item 统计密文中单字母频数
            \item 按出现的频率分组( 密文和明文都要统计: 1 8 5 7 5 )
            \item 分析字母组
            \begin{itemize}
                \item 单字母单词
                \item 双字母单词
                \item 三字母单词
                \item 四字母单词
            \end{itemize}
            \item 判断并翻译
        \end{enumerate}

        \item 同音代替密码
        \begin{itemize}
            \item 在同音代替密码中, 一个明文字母表中的字母a, 可以变换为若干个密文字母$f(a)$, 称为\hl{同音字母}. 从明文到密文的映射的形式是$f: A\rightarrow 2^C$, 其中A和C分别是\hl{明文字母表}和\hl{密文字母表}.

            例如 假定明文同音代替密码的密钥是一段短文

            Canada's large land mass and scattered populations makes efficient communication a neccessity. Extensive railway, road and other traspotation systems,
            as well as telephone, telegraph, and cable networks, have helped to link communities and have played a vital part in the country's development.

            将该短文机器中的各个单词编号如下:

            \begin{tabular}{|c|c|c|c|c|c|c|c|c|}
                \hline
                1 &2 &3 &4 &5 &6 &7 &8\\
                \hline
                Canada's &large &land &mass &and &scattered &populations &makes\\
                \hline
                9 &10 &11 &12 &13 &14 &15 &16\\
                \hline
                efficient &communication &a &neccessity &extensive &railway &road &and\\
                \hline
                17 &18 &19 &20 &21 &22 &23 &24\\
                \hline
                other &traspotation &systems &as &well &as &telephone &telegraph\\
                \hline
                25 &26 &27 &28 &29 &30 &31 &32 &32\\
                \hline
                and &cable &networks &have &helped &to &link &communities\\
                \hline
                33 &34 &35 &36 &37 &38 &39 &40\\
                \hline
                and &have &played &a &vital &part &in &The\\
                \hline
                41 &42 & & & & & &\\
                \hline
                country &development & & & & & &\\
                \hline
            \end{tabular}

            表中每个单词的首字母对应一个数字, 加密时, 可以用与某个字母相对应的任何一个数字来代替该字母

        \end{itemize}
    \end{itemize}

    \subsection{多表代替密码}
    \begin{itemize}
        \item 单表代替密码容易被攻破, 明文中的一个字母与密文中的一个字母一一对应, 明文中字母的统计特性在密文中可以反映出来.
        \item 多表代替密码采用多个密文字母表(提高代替密码强度), 使明文中的每一个字母都有多种可能的代替.

        构造由d个字符组成的密文字母表
        $$B_j=(b_{j0}, b_{j1}, b_{j2}, \cdots, b_{jn-1}), j=0,1, \cdots, d-1$$

        定义d个映射

        $$f_j: A\rightarrow B, f_j(a_i)=b_{ji}$$

        设明文
        $$M=(m_0, m_1, \cdots, m_{d-1}, m_d, \cdots)$$

        则相应的密文:
        $$C=[f_0(m_0), f_1(m_1), f_2(m_2), \cdots, f_{d-1}(m_{d-1}), f_d(m_d)]$$.

        \item 多表代替密码的\hl{密钥}就是\hl{d个映射}或者密文字母表
        \item 多表代替密码的种类有很多, 最著名的多表代替密码是16世纪的法国密码学者Vigenere使用过的Vigenere密码.
        此外, 还有\hl{游动钥密码}(running-key cipher)

        \item 弗吉尼亚密码简介
        \begin{enumerate}
            \item 设密钥$K=k_1k_2\cdots k_n$, 明文$M=m_1m_2\cdots m_n$, 加密变换为
            $$E_k(M)=c_1c_2\cdots c_n$$
            其中$c_i\equiv (m_i+k_i)\bmod26, i=1, 2,\cdots, n$.

            维吉尼亚密码的密钥可以周期性延长, 周而复始, 以至无穷. 即令
            $$K=k_1k_2k_3\cdots$$

            维吉尼亚方阵可以用来加密和脱密

            例如 $M=data security, k=best$, 试着写出对应的密文
            \begin{itemize}
                \item [1] 将M分解成长度为4的序列

                data secu rity

                \item [2] 加密过程, 以密钥字母$k_j(j=0, 1, 2, 3)$为行号, 明文字母$m_i(i=0, 1, 2, \cdots, 11)$, 在维吉尼亚方阵中依次得到对应密文字母.
                $$c=E_k(M)=EELT TIUN SMLR$$

                即维吉尼亚方阵中, b行d列的E为密文的第一个字母, e行a列的E唯密文的第二个字母, s行t列为密文的第三个字母, $\cdots$.
                或者利用公式$c_i=(m_i+k_i)\bmod 26, i=1, 2, \cdots, n$.

                $$
                \begin{aligned}
                    &c_1 = (m_1+k_1) \bmod 26 = (1+3) \bmod 26 = 4 = E\\
                    &c_2 = (m_2+k_2) \bmod 26 = (0+4) \bmod 26 = 4 = E\\
                    &\cdots \cdots\\
                    &c_{10} = (m_{10}+k_{10}) \bmod 26 = (1+3) \bmod 26 = 4 = E\\
                    &c_{11} = (m_{11}+k_{11}) \bmod 26 = (8+4) \bmod 26 = 12 = M\\
                    &c_{12} = (m_{12}+k_{12}) \bmod 26 = (24+19) \bmod 26 = 17 = R
                \end{aligned}
                $$

                \item [3] 脱密过程: 以密钥字母$k_j(j=0,1,2,3)$为行号, 在方阵中该行的密文字母$c_i(i=0,1,2, \cdots,11)$所在的列的列号即为明文字母.
                $$
                \begin{aligned}
                    &m_1 = (c_1-k_1) \bmod 26 = (4-1) \bmod 26 = 3 = D\\
                    &m_2 = (c_2-k_2) \bmod 26 = (4-4) \bmod 26 = 0 = A\\
                    &\cdots \cdots\\
                    &m_{10} = (c_{10}-k_{10}) \bmod 26 = (4-1) \bmod 26 = 3 = E\\
                    &m_{11} = (c_{11}-k_{11}) \bmod 26 = (12-8) \bmod 26 = 4 = M\\
                    &m_{12} = (c_{12}-k_{12}) \bmod 26 = (17-24) \bmod 26 = 19 = R
                \end{aligned}
                $$

            \end{itemize}

            \item 游动钥密码
            \begin{itemize}
                \item 游动钥密码是一种非周期性的Vigenere密码, 其\hl{密钥和明文信息一样长}, 且\hl{不重复}.

                例如 设$m=6$, 密钥字为cipher, 密钥k相对应的数字为$k=(2,8,15,7,4,1,7)$, 设明文是字符串this cryptosystem, 其加密过程如下

            \end{itemize}
            \begin{tabular}{|c|c|c|c|c|c|c|c|c|c|c|c|c|c|c|c|c|c|}
                \hline
                明文数据 &19 &7 &8 &18 &2 &17 &24 &15 &19 &14 &18 &24 &18 &19 &4 &12\\
                \hline
                密钥数据 &2 &8 &15 &7 &4 &1 &17 &2 &8 &15 &7 &4 &17 &2 &8 &15 &7\\
                \hline
                \hline
                密文数据 &21 &15 & 23 &25 &6 &8 &0 &23 &8 &21 &22 &15 &20 &1 &19 &19\\
                \hline
                密文字符串 &V &P &X &Z &G &I &A &X &I &V &W &P &U &B &T &T\\
                \hline
            \end{tabular}

            \item 对弗吉尼亚密码的分析
            \begin{itemize}
                \item 对弗吉尼亚密码进行分析, 首先要确定\hl{密钥的长度}, 记作m; 其次需要确定\hl{密钥字符串}. 一般用到两个技术: 一是Kaisiski测试, 二是重合指数验证
                \begin{itemize}
                    \item Kaisiski测试: 在密文中, 搜索长度至少为2的相同一对密文, 记下这两段开始点的距离; 如果可以获得几个这样距离$d_1, d_2, \cdots$, 则可以假设m是整出这些$d_i$的最大公因数

                    ---Kaisiski测试的基本出发点: 即两个相同的明文段将加密成相同的密文段, 如果在密文中观察到两个相同的长度至少为3的密文段, 则它们对应相同的明文串, 可以作为对该密文进行攻击的切入点.

                    \item 重合指数验证(所获得的m值)
                    \begin{itemize}
                        \item 重合指数: 假设$C=c_1c_2c_3\cdots c_n$是n个字符的串, C重合指数记为$IC(C)$, 定义为C中两个随即元素相同的概率.

                            假设用$n_A, n_B, \cdots, n_Z$分别表示为$A, B, \cdots, Z$在C中出现的频数; 以$C_n^2$种方式选择C中的两个元素, 其中$C_{n_A}^2$种方式选择两个元素都是A, $C_{n_B}^2$种方式选择两个元素都是B, $\cdots$, $C_{n_Z}^2$种方式选择两个元素都是Z.

                        则有
                        \begin{equation}
                            IC(C)=\frac{\sum_{\xi=A}^{Z} n_{\xi}(n_{\xi}-1)}{n(n-1)}
                        \end{equation}

                        其中$n_A, n_B, n_C, \cdots, n_Z$分别表示$A ,B, C, \cdots, Z$在C中出现的频数

                            随机英文字母序列Y的重合指数: 在Y中两个随机元素出现相同字母的概率为$\frac{1}{26}$, 即:
                            $$IC(Y)=\frac{1}{26}=0.0385$$

                            文献的X的重合指数: 设英文论文为X, 其中两个元素都是a的概率为$(0.0856)^2=0.0073274$, 两个元素都是b的概率为$(0.0139)^2=0.001932, \cdots$, 两个元素都是z的概率为$(0.0008)^2=0.00000064$.
                            若用$p_1$表示X中a出现的概率, 用$p_2$表示X中出现b的概率, $\cdots$, 用$p_{26}$表示X中出现z的概率, 则
                            $$IC(X)=\sum_{i=1}^{26} p_i^2 = 0.0687$$

                            重合指数验证方法: 设密文C为用Vigenere密码加密的得到的密文串$C=c_1c_2c_3\cdots c_n$. 假定使用的密钥词组长度为m, 使用的密钥词组长度为m, 即$K=k_1k_2k_3 \cdots k_m$, 而且各位由不同的字母组成.

                            \item 将密文C分成m行, 则每一行是单表代替密码, 不同的行由不同的密钥加密
                            \item 如果m是实际密钥的长度, 则同一行两个选定位置上有相同字母的概率为0.0687
                            \item 如果m不是实际密钥的长度, 则同一行两个选定位置上有相同字母的概率为0.0385

                            例如 从Vigenere密码中获得密文如下:

                            UFQUIUD\hl{DWF}HGLZARIHWLLWYYFSYYQATJP
                            FKMUXSSWWCSVFAEVWWGQCMVVSWFKUtB\hl{LLG}
                            ZFVITYOEIPA\hl{SJW}GGSJEPNSUETPTMPOPHZSF
                            DCXEPLZQWK\hl{DWF}XWTHASPWIUOVSSSFKWWL
                            CCEZWEUEHGVGLR\hl{LLG}WOFKWLUWSHEVWSTT
                            UARCWHWBVTGNITJRWWKCOTFGMILRQESKW
                            VMPFZMVEGEEPFODJQCHZIUZZMXKZBGJOTZ
                            AXCCMUMRS\hl{SJW}

                            [1] 统计得, 字符数共280个, 其中每个字母出现的频率如下表

                            \begin{tabular}{|c|c|c|c|c|c|c|c|c|c|c|c|c|}
                                \hline
                                A &B &C &D &E &F &G &H &I &J &K &L &M\\
                                \hline
                                9 &4 &10 &7 &14 &15 &14 &10 &11 &7 &9 &13 &10\\
                                \hline
                                \hline
                                N &O &P &Q &R &S &T &U &V &W &X &Y &Z\\
                                \hline
                                3 &7 &12 &9 &8 &20 &12 &14 &12 &27 &5 &6 &12\\
                                \hline
                            \end{tabular}

                            此时, $IC=0.0431$

                            [2] 作Kasiski试验, 列出重复的字符所在的位置及其间距:

                    \end{itemize}

                \end{itemize}
            \end{itemize}

        \end{enumerate}

    \end{itemize}

    \subsection{代数密码(Vernam加密算法)}
    \begin{itemize}
        \item 代数密码的加密方式如下:
        $$
        \begin{aligned}
            &M=m_1m_2m_3\cdots, m_i=0(or 1), i=1,2,\cdots\\
            &K=k_1k_2k_3\cdots, k_j=0(or 1), j=1,2,\cdots
        \end{aligned}
        $$

        即明文和密钥均用二元数组序列(二进制)表示

        \hl{加密过程如下}:
        $$c_i=m_i\oplus k_i, i=0, 1, \cdots$$

        即$c_i\equiv (m_i+k_i) \bmod 2, i=1,2,3 \cdots$

        \item 编制代数密码

        需要先把明文和密钥表示成二元序列, 再把它们按位模2相加即可.

        \item 把M和K转化为二元序列, 可按如下表所示(也可以按照其他方式转化).

        \begin{tabular}{|c|c||c|c||c|c|}
            \hline
            a &00000 &j &01001 &s &10010\\
            \hline
            b &00001 &k &01010 &t &10011\\
            \hline
            c &00010 &l &01011 &u &10111\\
            \hline
            d &00011 &m &01100 &v &10101\\
            \hline
            e &00100 &n &01101 &w &10110\\
            \hline
            f &00101 &o &01110 &x &10111\\
            \hline
            g &00110 &p &01111 &y &11000\\
            \hline
            h &00111 &q &10000 &z &11001\\
            \hline
            i &01000 &r &10001 & & \\
            \hline
        \end{tabular}

        解密过程为:
        $$m_i = c_i\oplus k_i i=0, 1, 2\cdots$$

        \item 可见, 代数密码的加密和解密非常简单, 特别适合计算机和通信系统的应用

        例如 明文cat, 密钥key, 用代数密码加密.

        解: $$M=00010 00000 10011$$
        $$K=01010 00100 11000$$
        则$C=M\oplus K=01100 00100 01011$

        密文为IEL.

        \item 代数密码属于序列密码, 其突出优点是加密变换和解密变换相同, 故使加密和解密软硬件实现极为简单, 加密和解密可共用一个软件模块或硬件电路.
        \item 若同用一个密钥重复使用或密钥本身包含重复, 则代数密码经不起"已知明文的攻击". 为增强代数密码的强度, 应该避免密钥重复. 可采用以下方式:
        \begin{itemize}
            \item 密钥是真正的随机序列.
            \item 密钥的长度大于或等于明文的长度.
            \item 一个密钥只用一次.
        \end{itemize}
    \end{itemize}

    \subsection{Hill密码}
    \begin{itemize}
        \item 基本思想: 将l个明文字母通过\hl{线性变换}把它们转换为l个密文字母. 解密时, 只要做一次逆变换即可. 在该算法中, 密钥即是\hl{变换举证}
        \begin{enumerate}
            \renewcommand\labelenumi{(\theenumi)}
            \item Hill加密算法
            设明文为: $M=m_1m_2\cdots m_l$, 密钥为$K=(k_{ij})_{l\times l}$
            $$C=KM(\bmod n)$$
            式中:
            \begin{equation}
                C=\left[ \begin{array}{c}{c_{1}} \\ {c_{2}} \\ {\vdots} \\ {c_{l}}\end{array}\right],
                M=\left[ \begin{array}{c}{m_{1}} \\ {m_{2}} \\ {\vdots} \\ {m_{l}}\end{array}\right],
                K=\left[ \begin{array}{cccc}{k_{11}} & {k_{12}} & {\cdots} & {k_{1 l}} \\ {0} & {k_{22}} & {\cdots} & {k_{2 l}} \\ {\vdots} & {\vdots} & {\ddots} & {\vdots} \\ {0} & {0} & {\cdots} & {k_{l l}}\end{array}\right]
            \end{equation}

            即加密变换中:
            $$E_k(M) = c_1c_2\cdots c_l = C$$

            其中
            \begin{equation}
                \left\{
                    \begin{aligned} &c_1=k_{11}m_1+k_{12}m_2+k_{13}m_3+\cdots +k_{1l}m_l (\bmod n)\\
                                    &c_2=k_{21}m_1+k_{22}m_2+k_{23}m_3+\cdots +k_{2l}m_l (\bmod n)\\
                                    &\cdots \cdots \\
                                    &c_3=k_{l1}m_1+k_{l2}m_2+k_{l3}m_3+\cdots +k_{ll}m_l (\bmod n)
                    \end{aligned}\right.
            \end{equation}

            一般取$n=26$

            例如 明文为hill, 采用$a\sim z$的26字母与数字$0\sim 25$对应. 若取$l=4, n=26$.

            密钥K为:

            $$K=\left[ \begin{array}{cccc}{8} &{6} &{9} &{5}\\ {6} &{9} &{5} &{10}\\ {5} &{8} &{4} &{9}\\ {10} &{6} &{11} &{4}\end{array} \right]$$

            解: 明文数字序列为 7 8 11 11, 即$M=(7 8 11 11)$

            于是有
            \begin{equation}
                \begin{aligned}
                    c_1&=k_{11}m_1+k_{12}m_2+k_{13}m_3+k_{14}m_4\\
                    &=8\times 7+6\times 8+9\times 11+5\times 11=258\equiv 24(\bmod 26)\\
                    c_2&=k_{21}m_1+k_{22}m_2+k_{23}m_3+k_{24}m_4\\
                    &=6\times 7+9\times 8+5\times 11+10\times 11=179\equiv 19(\bmod 26)\\
                    c_3&=k_{31}m_1+k_{32}m_2+k_{33}m_3+k_{34}m_4\\
                    &=5\times 7+8\times 8+4\times 11+9\times 11=242\equiv 8(\bmod 26)\\
                    c_4&=k_{41}m_1+k_{42}m_2+k_{43}m_3+k_{44}m_4\\
                    &=10\times 7+6\times 8+11\times 11+4\times 11=283\equiv 23(\bmod 26)\\
                \end{aligned}
            \end{equation}

            所以有C=(24 19 8 23)
            故密文为Y T I X

            例如 假设Hill密码加密使用密钥$K=\left[ \begin{array}{cc}{6} &{7}\\ {3} &{8} \end{array}\right]$, 试着对明文good加密.

            解: 明文数字序列为(6, 14, 14, 3), 把明文划分为两组(6, 14)和(14, 3), 加密过程如下:
            $$
            \begin{aligned}
                \left[ \begin{array}{c}{c_1}\\ {c_2}\end{array}\right]=K\left[ \begin{array}{c}{m_1}\\ {m_2} \end{array}\right]
                        =\left[ \begin{array}{cc}{6} &{7}\\ {3} &{8} \end{array}\right]\left[\begin{array}{c}{6}\\ {14} \end{array}\right]
                            &=\left[ \begin{array}{c}{134}\\ {130} \end{array}\right]\\
                                &=\left[ \begin{array}{c}{4}\\ {0} \end{array}\right](\bmod 26)
            \end{aligned}
            $$

            $$
            \begin{aligned}
                \left[ \begin{array}{c}{c_3}\\ {c_4}\end{array}\right]=K\left[ \begin{array}{c}{m_3}\\ {m_4} \end{array}\right]
                        =\left[ \begin{array}{cc}{6} &{7}\\ {3} &{8} \end{array}\right]\left[\begin{array}{c}{14}\\ {3} \end{array}\right]
                            &=\left[ \begin{array}{c}{105}\\ {66} \end{array}\right]\\
                                &=\left[ \begin{array}{c}{1}\\ {14} \end{array}\right](\bmod 26)
            \end{aligned}
            $$

            密文数据序列为(4 0 1 14), 则密文为EABO.
            \item Hill解密算法
            \begin{itemize}
                \item 一般来说, Hill密码能较好地抵抗频率分析. 采用"唯密文分析"很难攻破它, 但采用"已知明文的攻击"易被攻破.

                \hl{Hill解密算法为}:
                $$M=K^{-1}C(\bmod n)$$

                一般取$n=26$, 如果K有逆矩阵, 解密才是可能的.

                \item 一个实矩阵K有逆元的条件是: \hl{当且仅当其行列式非零}.
                \item $\because$Hill解密是在$Z_{26}$中进行的

                $\therefore$当且仅当$gcd\{detK, 26\}=1$时, 矩阵K有模26的逆元, 记为$(detK)^{-1}$.

                注: 若用记号$Z_m$表示集合, 则$Z_m=\{0, 1, \cdots, |m|-1\}$.

                \item $K^{-1}$的表达式为:
                $$K^{-1}=(detK)^{-1}K^*$$

                式中$K^*$称为K的\hl{伴随矩阵}, 其$(i, j)$单元为$A_{ji}$.

                \begin{equation}
                    K^*=\left[ \begin{array}{cccc}{A_{11}} &{A_{21}} &\cdots &{A_{l1}}\\ {A_{12}} &{A_{22}} &\cdots &{A_{l2}}\\ {A_{13}} &{A_{23}} &\cdots &{A_{l3}}\\ \vdots &\vdots &\ddots &\vdots\\ {A_{1l}} &{A_{2l}} &\cdots &{A_{ll}} \end{array}\right],
                        A_{ij}=(-1)^{i+j}detK_{ij}
                \end{equation}

                $K_{ij}(1\le i \le l, 1\le j \le l)$是从矩阵K中删除第i行和第j列后得到的矩阵.

                \item 对于$2\times 2$矩阵求逆有

                定理: 假设$A=(a_{ij})_{2\times 2}$, 且$detA$在$Z_{26}$中是可逆的, 则
                $$A^{-1}=(detA)^{-1}\left[ \begin{array}{cc}{a_{22}} &{-a_{12}}\\ {-a_{21}} &{a_{11}} \end{array}\right]$$

                例如 设$A=\left[ \begin{array}{cc}{6} &{7}\\ {3} &{8} \end{array}\right]$, 求$A^{-1}$

                    解: $detA=6\times 8 - 3\times 7=27\equiv 1(\bmod 26)$
                    $$
                    \because 1\times 1\equiv 1(\bmod 26)
                    \therefore (detA)^{-1}=1
                    $$

                    $$
                    \therefore
                    \begin{aligned}
                        A^{-1}&=\left[ \begin{array}{cc}{8} &{-7}\\ {-3} &{6} \end{array}\right](\bmod 26)\\
                            &=\left[ \begin{array}{cc}{8} &{19}\\ {23} &{6} \end{array}\right]
                    \end{aligned}
                    $$

                例如 假设Hill密码加密使用密钥$K=\left[ \begin{array}{cc}{6} &{7}\\ {3} &{8} \end{array}\right]$, 使用Hill密码对密文EABO解密

                    解: 密文数据序列为(4, 0, 1, 14)
                    $\because l=2, \therefore$把密文划分为两组(4, 0)和(1, 14)

                    求出K的逆矩阵$K^{-1}=\left[ \begin{array}{cc}{8} &{19}\\ {23} &{6}\end{array}\right]$, 则解密如下:

                    $$
                    \begin{aligned}
                        \left[ \begin{array}{c}{m_1}\\ {m_2}\end{array}\right]=K^{-1}\left[ \begin{array}{c}{c_1}\\ {c_2} \end{array}\right]
                                &=\left[ \begin{array}{cc}{8} &{19}\\ {23} &{6}\end{array}\right]\left[ \begin{array}{c}{4}\\ {0} \end{array}\right]
                                    =\left[ \begin{array}{c}{32} \\{92} \end{array}\right]\\
                                        &=\left[ \begin{array}{c}{6} \\{14} \end{array}\right](\bmod 26)
                    \end{aligned}
                    $$

                    $$
                    \begin{aligned}
                        \left[ \begin{array}{c}{m_3}\\ {m_4}\end{array}\right]=K^{-1}\left[ \begin{array}{c}{c_3}\\ {c_4} \end{array}\right]
                                &=\left[ \begin{array}{cc}{8} &{19}\\ {23} &{6}\end{array}\right]\left[ \begin{array}{c}{1}\\ {14} \end{array}\right]
                                    =\left[ \begin{array}{c}{274} \\{107} \end{array}\right]\\
                                        &=\left[ \begin{array}{c}{6} \\{14} \end{array}\right](\bmod 26)
                    \end{aligned}
                    $$

                    得到明文数字序列为(6, 14, 14, 3), 明文为good.
            \end{itemize}

            \item 关于Hill密码的已知明文攻击
            \begin{itemize}
                \item 假定分析者已知正在使用的l值, 且掌握了至少l个不同的l元组, 即
                $M_i=\left[ \begin{array}{c}{m_{1i}}\\ {m_{2i}}\\ \vdots\\ {m_{li}} \end{array}\right]$ 和
                    $C_i=\left[ \begin{array}{c}{c_{1i}}\\ {c_{2i}}\\ \vdots \\ {c_{li}} \end{array}\right]$

                满足
                $$C_i=E(M_i, K)=e_K(M_i), 1\le i\le l$$

                若定义两个$l\times l$矩阵: $M=(m_{ij})_{l\times l}, C=(c_{ij})_{l\times l}$, 则有矩阵方程
                $$C\equiv KM(\bmod 26)$$

                式中K是未知密钥K. 若M是可逆的, 则能计算出
                $$K\equiv CM^{-1}(\bmod 26)$$

                从而破译改密码体制. 但是, 若M不可逆, 则必须再试另外l个---密文对

                例如 假设明文worker利用$l=2$的Hill密码加密, 得到密文为QIHRYB, 求密钥.

                解: 明文worker 对应的数据序列为: (22, 14, 17, 10, 4, 17);

                密文QIHRYB 对应的数据序列(16, 8, 7, 17, 24, 1).

                因为l=2, 故将明文、密文按顺序分成三组, 即分别对应如下:
                $$\left( \begin{array}{c}{16}\\ {8} \end{array}\right)=K\left( \begin{array}{c}{22}\\ {14} \end{array}\right)(\bmod 26)$$
                $$\left( \begin{array}{c}{7}\\ {17} \end{array}\right)=K\left( \begin{array}{c}{17}\\ {10} \end{array}\right)(\bmod 26)$$
                $$\left( \begin{array}{c}{24}\\ {1} \end{array}\right)=K\left( \begin{array}{c}{4}\\ {17} \end{array}\right)(\bmod 26)$$

                从前面两个明文---密文对可以得到:
                $$\left[ \begin{array}{cc}{16} &{7}\\ {8} &{17} \end{array}\right]=K\left[ \begin{array}{cc}{22} &{17}\\ {14} &{10} \end{array}\right](\bmod 26)$$

                $
                    \because \left[ \begin{array}{cc}{22} &{17}\\ {14} &{10}\end{array}\right]=-18, (-18, 26)\neq 1
                    \therefore
                $重新考虑第二、第三组明文---密文对, 此时的到如下矩阵方程:

                $$\left[ \begin{array}{cc}{7} &{24}\\ {17} &{1} \end{array}\right]=K\left[ \begin{array}{cc}{17} &{4}\\ {10} &{17} \end{array}\right]$$

                $
                    \because \left| \begin{array}{cc}{17} &{4}\\ {10} &{17} \end{array}\right|=249, 249\times 7\equiv 1(\bmod 26)
                        \therefore (detA)^{-1}=7
                $容易算出
                $$
                \begin{aligned}
                    \left[ \begin{array}{cc}{17} &{-4}\\ {-10} &{17} \end{array}\right]=7\times \left[ \begin{array}{cc}{17} &{-4}\\ {-10} &{17} \end{array}\right]
                            &=\left[ \begin{array}{cc}{119} &{28}\\ {-70} &{119} \end{array}\right]\\
                                &\equiv \left[ \begin{array}{cc}{15} &{24}\\ {8} &{15} \end{array}\right](\bmod 26)
                \end{aligned}
                $$

                $\therefore$
                $
                \begin{aligned}
                    K=\left[ \begin{array}{cc}{7} &{24}\\ {17} &{1} \end{array}\right]\left[ \begin{array}{cc}{15} &{24}\\ {8} &{15} \end{array}\right]
                            &=\left[ \begin{array}{cc}{297} &{528}\\ {263} &{423} \end{array}\right]\\
                                &\equiv \left[ \begin{array}{cc}{11} &{8}\\ {3} &{7} \end{array}\right](\bmod 26)
                \end{aligned}
                $

                所得密钥K, 可以通过已知的明文---密文对进行验证.
            \end{itemize}
        \end{enumerate}
    \end{itemize}
\end{document}